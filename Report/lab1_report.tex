\documentclass[11pt]{article}
\usepackage[utf8]{inputenc}
\usepackage[T1]{fontenc}
\usepackage{graphicx}
\usepackage{grffile}
\usepackage{longtable}
\usepackage{wrapfig}
\usepackage{rotating}
\usepackage[normalem]{ulem}
\usepackage{amsmath}
\usepackage{textcomp}
\usepackage{amssymb}
\usepackage{capt-of}
\usepackage{hyperref}
\usepackage{minted}
\author{Student: Anna Sim AHS996 \\ Professor: Mohit Tiwari \\ TA: Austin Harris \\ Department of Electrical \& Computer Engineering \\ The University of Texas at Austin}
\date{\today}
\title{EE379K Enterprise Network Security Lab 1 Report}
\hypersetup{
 pdfauthor={Student: Anna Sim AHS996 \\ Professor: Mohit Tiwari \\ TA: Austin Harris \\ Department of Electrical \& Computer Engineering \\ The University of Texas at Austin},
 pdftitle={lab1 Report},
 pdfkeywords={},
 pdfsubject={},
 pdfcreator={},
 pdflang={English}}
\begin{document}

\maketitle
\section{Part 1}
\label{sec:part-1}
This first part is to create an interacting echo server and client in C. The server is to capitalize string input sent from the client. 
We are provided a python client and server as a model for our program to behave the same way. 
\subsection{Step 1: Echo Server}
\subsubsection{Build and Run Server and Client}
After writing the server and client files, run the following commands below in terminal to compile and run them.
It is crucial to start the server first, otherwise there is no connection.
\begin{code}
$ gcc server.c -o server
$ gcc client.c -o client
$ ./server
$ ./client
\end{code}

\subsection{Step 2: DOS Attack}
Before implementing a DOS attack, SYN cookies were first disabled on the host machine. This is done by setting \verb{net.ipv4.tcp_syncookies} to 0 in the \include{/etc/sysctl.conf} file.

\noindent To implement a DOS attack, the following command was used:
\begin{minted}{bash}
  $ sudo hping3 -S -w 64 -p 12000 --flood --rand-source 127.0.0.2
\end{minted}

\begin{figure}[htbp]
\centering
\includegraphics[width=.9\linewidth]{./rocc-encoding.png}
\caption{\label{fig:rocc-encoding}
The RoCC Accelerator Instruction Encoding}
\end{figure}
\section{Problem 3}
\label{sec:prob-3}
Nulla malesuada porttitor diam. Donec felis erat, congue non, volutpat at,
tincidunt tristique, libero. Vivamus viverra fermentum felis. Donec nonummy
pellentesque ante. Phasellus adip- iscing semper elit. Proin fermentum massa ac
quam. Sed diam turpis, molestie vitae, placerat a, molestie nec, leo. Maecenas
lacinia. Nam ipsum ligula, eleifend at, accumsan nec, suscipit a, ipsum. Morbi
blandit ligula feugiat magna. Nunc eleifend consequat lorem. Sed lacinia nulla
vitae enim. Pellentesque tincidunt purus vel magna. Integer non enim. Praesent
euismod nunc eu purus. Donec bibendum quam in tellus. Nullam cursus pulvinar
lectus. Donec et mi. Nam vulputate metus eu enim. Vestibulum pellentesque felis
eu massa.

\begin{minted}{c++}
  int main() {
    printf("Hello World");
    return 0;
  }
\end{minted}

\section{Conclusion}
\label{sec:conclusion}
Please provide feedback so we can improve the labs for the course. How many
hours did the lab take you? Was this lab boring? Did you learn anything? Is
there anything you would change? Feel free to put anything here, but leaving it
blank will result in the loss of points.

% \bibliography{bibliography}
% \bibliographystyle{ieeetr}
\end{document}
